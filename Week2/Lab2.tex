\documentclass{article}
\usepackage{Sweave}
\begin{document}
\title{R Demo on Prevalence}
\author{Elizabeth Mostofsky\footnote{STATA tutorial}
    \and Felipe Riveroll Aguirre\thanks{R Version \texttt{friveroll@gmail.com}}} 
\maketitle

\Sconcordance{concordance:Lab2.tex:Lab2.Rnw:%
1 1 1 1 0 9 1 1 9 12 1 1 2 4 0 2 2 4 0 1 2 1 6 5 0 1 2 3 0 2 2 1 0 1 1 %
3 0 1 2 1 1 1 2 10 0 1 1 10 0 1 1 11 0 1 2 6 1 1 2 4 0 2 2 10 0 1 1 9 0 %
1 1 10 0 1 2 12 1 1 2 1 0 1 1 3 0 1 2 1 1 1 5 41 0 1 3 3 1 1 4 37 0 1 2 %
30 1 1 2 1 0 4 1 3 0 1 2 1 1 1 2 47 0 1 2 30 1}


\section{Part I}


\begin{enumerate}
\item Objectives
\begin{enumerate}
\item Calculate the prevalence of smoking in the Framingham Data Set and interpret the results
\item Restrict an analysis to non-missing data
\item Create a 2 way table to examine changes in self-reported smoking status between visit 1 and visit 2
\end{enumerate}
\item Calculate the proportion of people at each visit that report current smoking (NA+) and the proportion of people at each visit that report current smoking among 
those with data on smoking status at that visit (NA-).   
\\In this data set, current smoking status us coded as ``0 = not current smoker, 1= current smoker''
\begin{enumerate}
\item Install the required package \texttt{Foreign} to read the dataset
\begin{Schunk}
\begin{Sinput}
> install.packages("foreign", dependencies = TRUE)
\end{Sinput}
\end{Schunk}
\item Load the library \texttt{Foreign}
\begin{Schunk}
\begin{Sinput}
> library("foreign")
\end{Sinput}
\end{Schunk}
\item Load and attach the dataset in a dataframe named data.
\begin{Schunk}
\begin{Sinput}
> data <- read.dta("C:\\Users\\felipillo\\
+                  Documents\\GitHub\\PH207x
+                  \\Data\\fhs.dta",
+                  convert.factors = TRUE ,
+                  missing.type = TRUE)
> attach(data)
\end{Sinput}
\end{Schunk}
\item Install and load the package epicalc
\begin{Schunk}
\begin{Sinput}
> install.packages("epicalc", dependencies = TRUE)
> library("epicalc")
\end{Sinput}
\end{Schunk}
\item Use \texttt{tab1} from \texttt{epicalc} to get one-way tabulation to get the frequency table for cursmoke 1,2 and 3.\\

\begin{Schunk}
\begin{Sinput}
> tab1(cursmoke1, graph=F, cum.percent = any(is.na(cursmoke1)))
\end{Sinput}
\begin{Soutput}
cursmoke1 : 
        Frequency Percent
No           2253    50.8
Yes          2181    49.2
  Total      4434   100.0
\end{Soutput}
\begin{Sinput}
> tab1(cursmoke2, graph=F, cum.percent = any(is.na(cursmoke2)))
\end{Sinput}
\begin{Soutput}
cursmoke2 : 
        Frequency   %(NA+) cum.%(NA+)   %(NA-) cum.%(NA-)
No           2203     49.7       49.7     56.1       56.1
Yes          1727     38.9       88.6     43.9      100.0
NA's          504     11.4      100.0      0.0      100.0
  Total      4434    100.0      100.0    100.0      100.0
\end{Soutput}
\begin{Sinput}
> tab1(cursmoke3, graph=F, cum.percent = any(is.na(cursmoke3)))
\end{Sinput}
\begin{Soutput}
cursmoke3 : 
        Frequency   %(NA+) cum.%(NA+)   %(NA-) cum.%(NA-)
No           2142     48.3       48.3     65.6       65.6
Yes          1121     25.3       73.6     34.4      100.0
NA's         1171     26.4      100.0      0.0      100.0
  Total      4434    100.0      100.0    100.0      100.0
\end{Soutput}
\end{Schunk}
NA+ proportion of people with missing data\\
NA- proportion of people among those with data
\end{enumerate}
\pagebreak
\item Calculate the proportion of people at each visit that report current smoking among those with data on smoking status at all 3 visits.
\begin{enumerate}
  \item We can create a dataframe excluding those with missing data (NA's)
\begin{Schunk}
\begin{Sinput}
> cursmokenotmiss <- na.exclude(data.frame(cursmoke1, cursmoke2, cursmoke3))
\end{Sinput}
\end{Schunk}
  \item Use \texttt{tab1} to get the proportions from the new dataframe cursmokenotmiss
\begin{Schunk}
\begin{Sinput}
> tab1(cursmokenotmiss$cursmoke1, graph=F)
\end{Sinput}
\begin{Soutput}
cursmokenotmiss$cursmoke1 : 
        Frequency Percent Cum. percent
No           1681    52.4         52.4
Yes          1525    47.6        100.0
  Total      3206   100.0        100.0
\end{Soutput}
\begin{Sinput}
> tab1(cursmokenotmiss$cursmoke2, graph=F)
\end{Sinput}
\begin{Soutput}
cursmokenotmiss$cursmoke2 : 
        Frequency Percent Cum. percent
No           1812    56.5         56.5
Yes          1394    43.5        100.0
  Total      3206   100.0        100.0
\end{Soutput}
\begin{Sinput}
> tab1(cursmokenotmiss$cursmoke3, graph=F)
\end{Sinput}
\begin{Soutput}
cursmokenotmiss$cursmoke3 : 
        Frequency Percent Cum. percent
No           2109    65.8         65.8
Yes          1097    34.2        100.0
  Total      3206   100.0        100.0
\end{Soutput}
\end{Schunk}

\end{enumerate}

\item What could explain the declining prevalence of smoking? 
\begin{enumerate}
  \item Over time, the prevalence of smoking is declining in the population
  \item Current smokers have a shorter life
  \item Several smokers choose not to participate in the 2nd and 3rd visits
\end{enumerate}

\item Calculate the change in smoking prevalence between the 1st and 2nd visit.
\begin{enumerate}
  \item Install and load the package \texttt{gmodels}
\begin{Schunk}
\begin{Sinput}
> install.packages("gmodels", dependencies = TRUE)
> library("gmodels")
\end{Sinput}
\end{Schunk}
\pagebreak
\item Use the command with to generate a 2 way frequency table with \texttt{CrossTable} from package \texttt{gmodels}, including missing values.
\begin{Schunk}
\begin{Sinput}
> with(data, CrossTable(cursmoke1, 
+                       cursmoke2, 
+                       missing.include=TRUE, 
+                       format="SPSS"))
\end{Sinput}
\begin{Soutput}
   Cell Contents
|-------------------------|
|                   Count |
| Chi-square contribution |
|             Row Percent |
|          Column Percent |
|           Total Percent |
|-------------------------|

Total Observations in Table:  4434 

             | cursmoke2 
   cursmoke1 |       No  |      Yes  |       NA  | Row Total | 
-------------|-----------|-----------|-----------|-----------|
          No |     1898  |      131  |      224  |     2253  | 
             |  541.582  |  635.078  |    4.022  |           | 
             |   84.243% |    5.814% |    9.942% |   50.812% | 
             |   86.155% |    7.585% |   44.444% |           | 
             |   42.806% |    2.954% |    5.052% |           | 
-------------|-----------|-----------|-----------|-----------|
         Yes |      305  |     1596  |      280  |     2181  | 
             |  559.461  |  656.043  |    4.154  |           | 
             |   13.984% |   73.177% |   12.838% |   49.188% | 
             |   13.845% |   92.415% |   55.556% |           | 
             |    6.879% |   35.995% |    6.315% |           | 
-------------|-----------|-----------|-----------|-----------|
Column Total |     2203  |     1727  |      504  |     4434  | 
             |   49.684% |   38.949% |   11.367% |           | 
-------------|-----------|-----------|-----------|-----------|
\end{Soutput}
\begin{Sinput}
> 
\end{Sinput}
\end{Schunk}

\end{enumerate}
\pagebreak
\item Calculate the change in smoking prevalence between the 1st and 2 nd visit among  those with data on smoking status at both visits.
\begin{Schunk}
\begin{Sinput}
> with(data, CrossTable(cursmoke1, 
+                       cursmoke2, 
+                       format="SPSS"))
\end{Sinput}
\begin{Soutput}
   Cell Contents
|-------------------------|
|                   Count |
| Chi-square contribution |
|             Row Percent |
|          Column Percent |
|           Total Percent |
|-------------------------|

Total Observations in Table:  3930 

             | cursmoke2 
   cursmoke1 |       No  |      Yes  | Row Total | 
-------------|-----------|-----------|-----------|
          No |     1898  |      131  |     2029  | 
             |  508.670  |  648.871  |           | 
             |   93.544% |    6.456% |   51.628% | 
             |   86.155% |    7.585% |           | 
             |   48.295% |    3.333% |           | 
-------------|-----------|-----------|-----------|
         Yes |      305  |     1596  |     1901  | 
             |  542.920  |  692.561  |           | 
             |   16.044% |   83.956% |   48.372% | 
             |   13.845% |   92.415% |           | 
             |    7.761% |   40.611% |           | 
-------------|-----------|-----------|-----------|
Column Total |     2203  |     1727  |     3930  | 
             |   56.056% |   43.944% |           | 
-------------|-----------|-----------|-----------|
\end{Soutput}
\end{Schunk}
\item Conclusions
\begin{enumerate}
  \item Smoking prevalence declined over time
  \begin{enumerate}
  \item Smokers are quitting
  \item Smokers have a shorter life
  \item Smokers are less likely to participate
\end{enumerate}
\item R can be used to
\begin{enumerate}
  \item Restrict an analysis to non-missing data
  \item Create a 2 way table to cross-classify two nominal variables
\end{enumerate}

\end{enumerate}

\end{enumerate}
\pagebreak
\section{Part II}
\begin{enumerate}
  \item Objectives
  \begin{enumerate}
  \item Create an ordinal variable from continuous data 
  \item Calculate the prevalence of CHD for different levels of smoking at visit 1
\end{enumerate}
\item Calculate the prevalence of coronary heart disease (CHD) at visit 1 by categories of cigarettes per day\\
\\ ``PREVCHD is defined as pre-existing angina pectoris, myocardial infarction (hospitalized, silent or unrecognized), or coronary insufficiency (unstable angina) 0 = Free of disease, 1 = Prevalent disease''
\begin{enumerate}
  \item Create 4 categories of cigarette packs per day ( 0 , 1-20 , 21-40, \geq 41 ).
  \\ Since the values reflect, a particular ordering, it is an ordinal variable.

\begin{Schunk}
\begin{Sinput}
> data$packs1 <- NA # initialize packs1
> data$packs1 [data$cigpday1==0] <- 0
> data$packs1 [data$cigpday1>=1 & data$cigpday1 <= 20] <- 1
> data$packs1 [data$cigpday1>=21 & data$cigpday1 <= 40] <- 2
> data$packs1 [data$cigpday1>=41 & !is.na(data$cigpday1)] <- 3
\end{Sinput}
\end{Schunk}
\pagebreak
\item Use CrossTable to get a 2 way table from packs1 and prevchd1
\begin{Schunk}
\begin{Sinput}
> with(data, CrossTable(packs1, prevchd1, format="SPSS"))
\end{Sinput}
\begin{Soutput}
   Cell Contents
|-------------------------|
|                   Count |
| Chi-square contribution |
|             Row Percent |
|          Column Percent |
|           Total Percent |
|-------------------------|

Total Observations in Table:  4402 

             | prevchd1 
      packs1 |       No  |      Yes  | Row Total | 
-------------|-----------|-----------|-----------|
           0 |     2145  |      108  |     2253  | 
             |    0.049  |    1.073  |           | 
             |   95.206% |    4.794% |   51.181% | 
             |   50.938% |   56.545% |           | 
             |   48.728% |    2.453% |           | 
-------------|-----------|-----------|-----------|
           1 |     1606  |       65  |     1671  | 
             |    0.035  |    0.777  |           | 
             |   96.110% |    3.890% |   37.960% | 
             |   38.138% |   34.031% |           | 
             |   36.483% |    1.477% |           | 
-------------|-----------|-----------|-----------|
           2 |      383  |       15  |      398  | 
             |    0.014  |    0.298  |           | 
             |   96.231% |    3.769% |    9.041% | 
             |    9.095% |    7.853% |           | 
             |    8.701% |    0.341% |           | 
-------------|-----------|-----------|-----------|
           3 |       77  |        3  |       80  | 
             |    0.003  |    0.064  |           | 
             |   96.250% |    3.750% |    1.817% | 
             |    1.829% |    1.571% |           | 
             |    1.749% |    0.068% |           | 
-------------|-----------|-----------|-----------|
Column Total |     4211  |      191  |     4402  | 
             |   95.661% |    4.339% |           | 
-------------|-----------|-----------|-----------|
\end{Soutput}
\end{Schunk}

\end{enumerate}
\pagebreak
\item What could explain the higher prevalence of CHD among non-smokers compared to 
those who smoke 1 or more cigarettes per day?
\begin{enumerate}
  \item High incidence, Long duration
  \item Cross-sectional data is susceptible to reverse causation
  \item Other common suspects
  \begin{enumerate}
  \item Bias
  \item Confounding 
  \item Chance
\end{enumerate}

\end{enumerate}

\item Conclusions
\begin{enumerate}
  \item R can be used to create an ordinal variable based on continuous data. 
  \item CHD prevalence was lower among people with higher levels of smoking.
  \item Prevalence is a function of incidence and duration. 
  \item In addition to a causal effect of exposure on disease risk, there are several 
alternative explanations for observing an association between two factors of interest.
\end{enumerate}

\end{enumerate}


\end{document}

