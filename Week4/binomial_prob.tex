%&latex
\documentclass{article}

\usepackage[english]{babel}
\begin{document}

%+Title
\title{Binomial Probability R and STATA Commands}
\author{Felipe Riveroll Aguirre}
\date{\today}
\maketitle
%-Title

\begin{enumerate}
\item 
\(P(X\leq k)=P(k=0)+...+P(X=k)\)\\
\textit{STATA: }\texttt{binomial(n,k,p)}\\
\textit{R:} \texttt{pbinom(q, size, prob, lower.tail = TRUE)}

\item

\(P(X>k) = 1 - P(X\leq k)\)\\
\textit{STATA: }\texttt{1- binomial(n, k, p)} or \texttt{binomialtail(n,k+1,p)}\\
\textit{R: }\texttt{pbinom(q, size, prob, lower.tail = FALSE)}

\item 
\(P(X<k) = P(X\leq (k-1))\)\\
\textit{STATA:} \texttt{binomial(n, k-1, p)} or \texttt{1 - binomialtail(n, k, p)}\\
\textit{R:} \texttt{pbinom(q-1, size, prob, lower.tail = TRUE)}

\item
\(P (X\geq k) = P(X>(k-1))=1-P(X<k)\)\\
\textit{STATA:} \texttt{binomialtail(n,k, p)}\\
\textit{R: }\texttt{pbinom(q-1, size, prob, lower.tail = FALSE\\ }

\end{enumerate}


\begin{center}\fbox{In R \(k\) refers to the \(q\) argument in \texttt{pbinom}}\end{center}


\end{document}


